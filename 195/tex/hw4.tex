\documentclass{article}
\linespread{2.0}
\usepackage{amsfonts}
\usepackage{mathtools}
\usepackage{systeme}
\usepackage{extpfeil}
\usepackage{graphicx} % Required for inserting images

\begin{document}

1. Show that a composition of natural transformations is a natural transformation.

Let $\mathcal{C}, \mathcal{D}, \mathcal{E}$ be categories. 

Let $F,G: \mathcal{C} \rightrightarrows \mathcal{D}$ and $K,L: \mathcal{D} \rightrightarrows \mathcal{E}$ be functors.

Let $\alpha: F \Rightarrow G$ and $\beta: K \Rightarrow G$ be natural transformations.

Since the composition of functors is functor, we can define the functor $FK, GL: \mathcal{C} \rightrightarrows \mathcal{E}$. We WTS the composition of natural transformations of these functors, $\alpha\beta: FK \Rightarrow GL$, is a natural transformation. 

Let $c, d$ be objects and a morphism $f: c\to d$ in $\mathcal{C}$. Define the component of $\alpha\beta$ at $c$ as $\alpha\beta_c\coloneqq \beta_{cF}$. Since $\beta: K \Rightarrow G$ is a natural transformation, its component at $cF$ is $\beta_{cF}: cFK \to cGL$, which are the same image of $c$ under $FK$ and $GL$. 

Moreover, the morphism $f$, under $FK$ and $GL$, becomes $f^{FK}$ and $f^{GL}$, each together with components of $\alpha\beta$ as defined above, commutes since $\beta$ and $\alpha$ are natural transformations.

Thus, $\alpha\beta$ is a natural transformation. $\square$

2. 

\end{document}

\documentclass{article}
\linespread{2.0}
\usepackage{mathtools}
\usepackage{amsfonts}
\usepackage{graphicx} % Required for inserting images

\title{c2s4}

\begin{document}
1.
Let $\{ x_n \}^\infty_{n=1}$ and $\{ y_n\}^\infty_{n=1}$ such that $\lim_{n \rightarrow\infty}y_n = 0$. Suppose that for all $k \in \mathbb{N}$ and for all $m \geq k$, we have

$\hspace{5cm} |x_m - x_k| \le y_k$

Show that $\{ x_n \} ^\infty_{n=1}$ is Cauchy. 

Proof:

Let $\epsilon >0$ be given 

Since $\lim_{n \rightarrow\infty}y_n = 0$, we have that $\exists N \in\mathbb{N}$ such that $\forall k > N$, we have 

$\hspace{5cm}|y_k| < \epsilon$

Observe that:

$\hspace{5cm} |x_m - x_k| \le y_k \le |y_k| <\epsilon$

So, $\{ x_n \}^\infty_{n=1}$ is Cauchy. $\square$


3. Prove or disprove: A Cauchy sequence can have an unbounded subsequence. 

Proof: 

Let $\{x_n\}$ be a Cauchy sequence. So $\{x_n\}$ is bounded, so $\exists M \in \mathbb{R}$ such that $\forall n\in \mathbb{N}, x_n \le M$. So any subsequence $\{ x_{n_k}\}$ will be bounded by $M$. So, there is not a subsequence of $\{x_n\}$ that is unbounded. $\square$


2. Suppose $|x_n-x_k| \le \frac{n}{k^2}$ for all $n$ and $k$. Show that $\{x_n\}^\infty_{n=1}$ is Cauchy. 

Proof: 

Let $\epsilon > 0 $ be given. 

Let $ N \in \mathbb{N}$ such that $ N > \frac{1}{\epsilon}$. Then $\forall n \geq N$, and any $k \in \mathbb{N}$, we have the following: 

$\hspace{2.5cm} \epsilon > \frac{1}{n} = \frac{n}{n^2} > \frac{n}{(n+k)^2}\hspace{3cm} (1)$

Since $|x_n-x_k| \le \frac{n}{k^2}$ for all $n$ and $k$, it is also true when $k = n+k$, then

$\hspace{2cm} (1) \Leftrightarrow \epsilon > \frac{n}{(n+k)^2} > |x_n -x_{n+k}|$

Since $n$ and $n+k$ is any number larger than $N$, we have that $\{x_n\}^\infty_{n=1}$ is Cauchy. $\square$
\end{document}

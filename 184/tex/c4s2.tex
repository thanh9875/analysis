\documentclass{article}
\linespread{2.0}
\usepackage{amsfonts}
\usepackage{mathtools}
\usepackage{systeme}
\usepackage{graphicx} % Required for inserting images

\begin{document}
1. Suppose $f: \mathbb{R} \rightarrow \mathbb{R}$ is a function such that $|f(x) - f(y)| \le |x-y|^2$ for all $x, y \in \mathbb{R}$. Show that $f(x) = C$ for some constant $C$.

Let $c \in \mathbb{R}$.

Note that: $\frac{|f(x)-f(c)|}{|x-c|} \le \frac{|x-c|^2}{|x-c|} = |x-c|$, then $\newline \lim_{x\rightarrow c} \frac{|f(x)-f(c)|}{|x-c|} \le \lim_{x\rightarrow c} |x-c| = 0$. So, $\lim_{x\rightarrow c} \frac{f(x)-f(c)}{x-c} = f'(c)= 0$ for all $c \in \mathbb{R}$. So $f$ is differentiable on $\mathbb{R}$.

Let $a,b \in \mathbb{R}$ such that $a < b$, then the interval $[a,b] \in \mathbb{R}$. So, by Mean Value Theorem, $\exists n \in (a,b)$ such that $f(b)-f(a)=f'(n)(b-a)$. Since $f'(n) = 0$ for all $n \in \mathbb{R}$, this follows $f(a) = f(b) = C$ for all $a, b \in \mathbb{R}$ and some constant $C$. $\square$ 

$\newline$

2. Suppose $f: S \rightarrow \mathbb{R}$ is a differentiable function and $f'$ is bounded. Prove that $f$ is a Lipschitz continuous function. 

Let $x, y \in \mathbb{R}$ and $x < y$.

Then the interval $[x,y] \in \mathbb{R}$ and by Mean Value Theorem, $\exists c \in (x,y)$ such that $|\frac{f(y)-f(x)}{y-x}| = |f'(c)|$. Since $f'$ is bounded, $\forall c \in \mathbb{R}, |f'(c)| \le M$ for some $M \in \mathbb{R}$. Finally, we have $|\frac{f(y)-f(x)}{y-x}| = |f'(c)| \le M$, thus $|f(y)-f(x)| \le M|y-x|$ for all $x,y$ in $\mathbb{R}$. So $f$ is a Lipschitz continuous function.$\square$

$\newline$

3. Suppose $f: (a,b) \rightarrow \mathbb{R}$ and $g: (a,b) \rightarrow \mathbb{R}$ are differentiable functions such that $f'(x) = g'(x)$ for all $x \in (a,b)$. Prove that there exists a constant $C$ such that $f(x) = g(x) + C$.

Let $h: (a,b)\rightarrow \mathbb{R}$ and $h(x) \coloneqq f(x)-g(x)$. Since both $f$ and $g$ are differentiable, $h$ is also differentiable and $h'(x) = f'(x)-g'(x) = 0$ for all $x \in (a,b)$.

Then by Proposition 4.2.6, $h(x) = C$ for some $C \in \mathbb{R}$, thus $f(x) - g(x) = C$. $\square$

$\newline$

4. Suppose $a, b \in \mathbb{R}$ and $f: \mathbb{R} \rightarrow \mathbb{R}$ is differentiable, $f'(x) = a $ for all $x \in \mathbb{R}$, and $f(0) = b$. Find $f$ and prove its uniqueness.

Let $x, y \in \mathbb{R}$ and $x < y$.

Then the interval $[x,y] \in \mathbb{R}$ and by Mean Value Theorem, $\exists c \in (x,y)$ such that $\frac{f(x)-f(y)}{x-y} = f'(c) = a$. This follows $f(y) - f(x) = a(y-x)$. Since this holds for all $x, y \in \mathbb{R}$, it is also true for $x=0$. Then the equation becomes $f(y) - b = ay$, or $f(y) = ay +b$. 

Let $h$ be a differentiable function satisfies all conditions for $f$, and $\newline h(x) \not= ax+b$. Note that for all $x \in \mathbb{R}$, $f'(x) = h'(x) = a$, then by Problem 3, $f(x) = h(x) + C$ for some constant $C$. Substituting $x=0$ shows that $C = 0$, so $f(x)$ is unique. $\square$

\end{document}

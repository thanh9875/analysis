\documentclass[12pt]{article}
\usepackage{fullpage}
\usepackage{graphicx}
\usepackage{sidecap}
\usepackage{algorithmic}
\usepackage{algorithm2e}
\usepackage{bbm}
\usepackage{amssymb}
\usepackage{amsmath}
\usepackage{amsfonts}
\usepackage{amsthm}
\usepackage{yhmath}
\usepackage{siunitx}
\usepackage[mathscr]{euscript}
\usepackage{enumerate}
\usepackage{enumitem}
\usepackage{mathtools}
\usepackage[hmargin=1in,vmargin=1in]{geometry}
\usepackage{graphicx}
\usepackage{subfigure}
\usepackage{setspace}
\usepackage{systeme}
\usepackage{mathrsfs}
\linespread{1.4142}
\graphicspath{ {./images/} }

\newcommand{\NN}{\mathbb{N}}
\newcommand{\Z}{\mathbb{Z}}
\newcommand{\Q}{\mathbb{Q}}
\newcommand{\R}{\mathbb{R}}
\newcommand{\C}{\mathbb{C}}
\newcommand{\curve}{\mathcal{C}}
\newcommand{\prob}[1]{\textbf{Problem #1:}}
\DeclarePairedDelimiter{\ceil}{\lceil}{\rceil} %for ceiling function
\DeclarePairedDelimiter{\floor}{\lfloor}{\rfloor} %for flooring function
%proofs envinronment
\newenvironment{proofs}{%
    \par\noindent\textit{Proof:} % Bold "Proof:" at the beginning
    \par\noindent
}{%
    \hfill$\qed$ % qed symbol at the end
    \par\noindent
}

%multiset notation
\def\multiset#1#2{\ensuremath{\left(\kern-.3em\left(\genfrac{}{}{0pt}{}{#1}{#2}\right)\kern-.3em\right)}}


\begin{document}
\prob{1} Determine how many different lottery tickets can be made?

\begin{enumerate}[label=(\alph*)]
    \item A ticket of selecting six numbers in $[16]$ where repitition is allows and order matters.
    \item A ticket of selecting five numbers in $[25]$ where order does not matter.
    \item A ticket of selecting four distinct numbers in $[18]$ and order matters.
\end{enumerate}

\prob{2} Binary strings are words whose characters are either $0$ or $1$. For each $k,n \in \mathbb{N}$, with $k \le n$, how many binary strings of length $n$ have exactly $k$ ones? How many have exactly $k$ zeros?

\prob{3} How many $n-$digit palindromes can be created using $1-9$ and repitition is allowed?

\prob{4} How many subsets of $[20]$ that have:

\begin{enumerate}[label=(\alph*)]
    \item have the smallest number be $4$ and the largest be $15$?
    \item contains no even numbers?
    \item have size $10$ and do not contain any number larger than $17$?
\end{enumerate}

\prob{5} Enumerate and count all $4-$lists $(\ell_1, \ell_2, \ell_3, \ell_4)$ such that $\ell_1, \ell_2, \ell_3, \ell_4 \in \Z$ and $1 \le \ell_1 < \ell_2 < \ell_3 < \ell_4 \le 6$.

\prob{6} How many $6$-letter words do not simultaneously begin and end with a vowel?

\prob{7} How many $4$-permutations of $[10]$ have maximum element equal to $6$? How many have the maximum element of at most $6$?

\prob{8} Find the number of $4-$list of the form $(x_1, x_2, x_3, x_4)$ such that $x_i \in \mathbb{N}$ for each integer $1 \le x_i \le 4$ such that $x_1 + x_2 + x_3 + 4x_4 = 5$

\prob{9} How many binary strings of length $n$ have at least one $0$ and at least one $1$ simultaneously?

\prob{10} How many nonempty subsets of $[10]$ have the product of their elements even?

\prob{11} In $n \in \mathbb{Z}^+$ how many are in the less-than relation on $[n]$? How many are in the less-than-or-equal-to relation?

\prob{12} Show that the number of subsets of $[n]$ is equal to the number of binary strings of length $n$.

\prob{13} Suppose $n \in \mathbb{N}$. Show that the number of even-size subset equals to the number of odd-size subsets.

\prob{14} Define $(n)^k$ and show that $\multiset{n}{k} = \frac{(n)^k}{k!}$.

\prob{15} Suppose an ice-cream shop makes three different types of ice-cream cones. How many different ways are there to buy $9$ cones and eat one of them?

\prob{16} Suppose $A \subseteq \R$ is a finite set and $n,k \in \mathbb{Z}^+$

\begin{enumerate}[label=(\alph*)]
    \item Prove that the number of solutions to $a_1 \le a_2 \le \dots \le a_k$ with $a_i \in A$ for each $i \in [k]$ is $\multiset{|A|}{k}$.
    \item Determine the number of solutions to $1 \le a_1 \le a_2 \le \dots \le a_k \le n$ where $a_i$ is an odd integer in $[n]$ for each $i \in [k]$.
\end{enumerate}

\prob{17} Suppose that $n \in \mathbb{Z}^+$ Use the sum principle to prove that the total number of compositions of $n$ is $2^{n-1}$.

\prob{18} Determine the number of ordered integer solutions to $x_1 + x_2 + x_3 + x_4 = 26$ such that $1 \le x_4 \le 3$ and $x_i \geq i$ for each $i\in [3]$.

\prob{19} Suppose $k, n \in \mathbb{Z}^+$ with $m \le n$ and $k \le \floor{\frac{m-n}{2}}$. In $P_n$, how many ways can we color $k$ non-adjacent nodes red and at most $m$ nodes green?

\prob{20} Suppose $n$ is a positive integer. Determine the maximum and mimimum possible size of an equivalence relation on $[n]$?

\prob{21} Suppose $n \geq 3$ is an integer. Determine the number of ways to seat $n$ people at a round table where two arrangements are considered equivalent whenever everyone at the table has the asme exact same set of neighbors.

\prob{22} How many ways are there to seat five professors and five students around a round table with $10$ chairs such that the seating alternates student-professors?

\prob{23} Use the equivalence principle to prove the formula $(n)_k = \frac{n!}{(n-k)!}$

\prob{24} How many ways can we split a group of $10$ people into two groups of size $3$ and one group of size $4$?

\textbf{Bonus:} Suppose $j, k \in \mathbb{Z}+$ with $n=j\cdot k$. Use the combinatorial proof to show that $$\frac{1}{j!} \prod\limits^{j-1}_{i=0} \binom{n-ik}{k} = \frac{n!}{(k!)^j j!}$$ Hints: try to justify the left hand side, and then generalize the argument for arbitrary value of $j$ and $k$.

\prob{25} Suppose $n \in \mathbb{Z}^+$ and let $S \in \binom{[2n]}{n+1}$. Prove that there exist two element in $S$ whose sum is $2n+1$.

\prob{26} Consider any five points in the plane that have integer coordinates.

\begin{enumerate}[label=(\alph*)]
    \item Prove that there are two points such that the midpoint of the line segment joining those two points also has integer coordinates.
    \item Show the counterexample for $4$ points.
\end{enumerate}

\prob{27} Show a counterexample of $n^2$ sequence of distinct real numbers can only have at most $n$ increasing or decreasing subsequence.

\prob{28} Let $n \in \mathbb{Z}^+$ and $k \in \mathbb{N}$. Suppose exactly $k$ nodes in $P_n$ are colored blue. Prove that there exists a blue node and a non-blue node that are at least $\ceil{\frac{n-k}{k+1}}$ steps away from each other.

\prob{29} Consider the possible functions $f: [7] \to [9]$.

\begin{enumerate}[label=(\alph*)]
    \item How many have $f(2) = 4$ and $f(3) \neq  8$?
    \item How many have $f(3) \neq 8$ and are one-to-one?
    \item How many have $f(i)$ even for all $i \in [7]$ and are one-to-one?
    \item How many have $f(i)$ even for all $i \in [7]$?
\end{enumerate}

\prob{30} Let $n \in \mathbb{Z}^+$. Determine the number of surjective functions $f: [n] \to [4]$

\prob{31} Suppose $n, j, k \in \mathbb{Z}^+$ with $k,j \le n$. Give a combinatorial proof that $$(n)_k = (n-j)_{(k-j)} \cdot (n)_j$$

\prob{32} Suppose $k$ and $n$ are integers with $1 \le k \le n$. Give combinatorial proof that $$ (n)_k = \sum\limits^k_{n=k} k(j-1)_{k-1}$$

\prob{33} Prove that, for $k,n\in \mathbb{Z}^+$, $\multiset{n}{k} = \multiset{n-1}{k} + \multiset{n}{k-1}$

\prob{34} Prove that, for $n \geq 1$, $3^n = \sum\limits^{n}_{k=0} \binom{n}{k} 2^{n-k}$

\prob{35} Prove that, for $m, n \in \mathbb{Z}^+$ and $m \le n$, $\sum\limits_{j=m}^n \binom{j}{m} = \binom{n+1}{m+1}$.

\prob{36} Prove that the number of even-sized subset equals the number of odd-sized subset.


\textbf{Bonus:} Prove that, for $n \in \mathbb{Z}^+$ and $x,y \in \R$, $$(x+y)_n = \sum\limits^n_{k=0} \binom{n}{k} (x)_k (y)_{n-k}$$.

\prob{38} Consider the letters in the wod INVISIBILITIES

\begin{enumerate}[label=(\alph*)]
    \item How many distinct ways are there to arrange $14$ letters in the word?
    \item How many distinct ways are there if no tow I's are allowed to be adjacent?
    \item How many distinct arrangements are there if the longest number of consecutive I's in a row is $4$?
\end{enumerate}

\prob{39} Let $n \in \Z^+$ and consider the sequence of non-negative integers $x_1, x_2, \dots, x_n$. If there are some people are placed into teams so that for each $i \in [n]$, there are exactly $x_i$ teams of size $i$, then the number of ways to create the teams is $$\frac{n!}{\prod\limits^n_{i=1} (i!)^{x_i} x_i!}$$.

\prob{40} Determine $\lim\limits_{k \to \infty} \frac{1}{k} \binom{-1}{k}$.

\prob{41} Prove that $\sum\limits^n_{k=m} \binom{n}{k} \binom{k}{m} = \binom{n}{m} 2^{n-m}$.

\prob{42} Let $A$ be a $10$ element set. How many equivalence relations are there on $A$? How many have exactly $8$ equivalence classes?

\prob{43} How many onto functions form $[9]$ to $[7]$ have only one element mapped to $7$?

\prob{44} Give a formula for the nubmer of ways to place $n$ distinguishable boxes so that at least one box stays empty.

\prob{45} Give a combinatorial proof that for integer $k,n \geq 1$, $S(n,k) = \sum\limits^{n-1}{i=0} \binom{n-1}{i} S(i, k-1)$. 

\prob{46} Use the face that $P(n,k) = P(n-1, k-1) + P(n-k, k)$ for integers $n, k$ to prove that for all integer $n \geq 3$, $P(n,k) = \floor{\frac{n}{2}}$. 

\prob{47} For each $n \in \Z^+$, let $P(n)$ be the nubmer of partitions of $n$. 
    \begin{enumerate}[label=(\alph*)]
        \item Give a combinatorial proof that $P(n) = \sum\limits^n_{k=1} P(n,k)$
        \item Gibve a bijective proof that $P(n) = P(2n, n)$ for each $n \in \mathbb{N}$.
    \end{enumerate}

\prob{48} Let $z_i$ be a partitions of $n$ into $k$ parts in non-increasing order. Show how to compute the conjugate of the partitions in terms of $z_i$

\prob{49} Suppose $P(n)$ is odd for $n \in \Z^+$. Show that at least one partitions of $n$ is self-conjugate. 

\prob{50} How many integer in $[100]$ are not divisible by $4$, $6$, and $7$?

\prob{51} Suppose that in an inclusion-exclusion problem, there exits a functions $f$ such that for any subset $J \subseteq P$ with $|J| = j$, $N_{\geq} (J) = f(j)$. Prove that $$N_{=} (\varnothing) = \sum\limits^n_{j=0} \binom{n}{j} (-1)^j f(j)$$. 

\prob{52} Use inclusion-exclusion to prove that the number of partitions of an $n-$set into $k$ parts is $$S(n,k) = \frac{1}{k!} \sum\limits^k_{j=0} \binom{k}{j} (-1)^j (k-j)^n$$

\prob{53} Derive an identity for $\binom{n}{k}$ via inclusion-exclusion by counting the $k-$multisets of $[n]$ is which each element of $[n]$ appears at most once. 

\prob{54} Suppose $n,k \in \Z^+$ and suppose $j$ is an integer with $0 \le j \le k$,
\begin{enumerate}[label=(\alph*)]
    \item How many $k-$lists taken from $n$ have exactly $j$ entries that are $n$?
    \item Use part $(a)$ to partition the $k-$lists taken from $[n]$ and get the formula for $n^k$.
    \item Prove the formula using the binomial theorem
\end{enumerate}

\prob{55} Prove that, for all $n,k \in \Z^+$, $\sum\limits^n_{j=1} \multiset{j}{k-1} = \multiset{n}{k}$. 

\prob{56} Prove that, for all $n,k \in \Z^+$, $\binom{kn}{2} = k\binom{n}{2} + n^2 \binom{k}{2}$.

\prob{57} Prove the following identities:
\begin{enumerate}[label=(\alph*)]
    \item $\binom{20}{8} \binom{8}{5} \binom{5}{3} = \binom{20}{3} \binom{17}{2} \binom{15}{4}$
    \item For all positive integer $n \geq k \geq j$, $\binom{n}{k} \binom{k}{j} = \binom{n}{j} \binom{n-j}{k-j}$
\end{enumerate}

\end{document}

\begin{comment}

to delete files: latexmk -c

to compile: pdflatex (filename).tex

to add a pdf
\begin{figure}[!htbp]
    \centering
    \includegraphics[width=1 \linewidth]{{filename}.pdf}
    \caption{Data sheet}
\end{figure}

\end{comment}

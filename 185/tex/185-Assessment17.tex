\documentclass[12pt]{article}
\usepackage{fullpage}
\usepackage{graphicx}
\usepackage{sidecap}
\usepackage{algorithmic}
\usepackage{algorithm2e}
\usepackage{bbm}
\usepackage{amssymb}
\usepackage{amsmath}
\usepackage{amsfonts}
\usepackage{amsthm}
\usepackage{yhmath}
\usepackage[mathscr]{euscript}
\usepackage{enumerate}
\usepackage[hmargin=1in,vmargin=1in]{geometry}

\begin{document}

Let $\mathbb{X}\subseteq\mathbb{R}$, $\mathscr{M}$ is a $\sigma$-algebra, and $\mu$ is a measure on $(\mathbb{X},\mathscr{M})$.  Prove the following:
\begin{enumerate}
\item If $E,F\in\mathscr{M}$, and $E\subseteq F$, then $\mu(E)\leq\mu(F)$.
\item If $\{E_j\}_{j=1}^\infty\subseteq\mathscr{M}$, then $\mu\left(\bigcup_{j=1}^\infty E_j\right)\leq\sum_{j=1}^\infty\mu(E_j)$
\item If $\{E_j\}_{j=1}^\infty\subseteq\mathscr{M}$ and $E_1\subseteq E_2\subseteq E_3\subseteq\cdots$, then $\mu\left(\bigcup \limits_{j=1}^\infty E_j\right)=\lim\limits_{j\rightarrow\infty}\mu(E_j)$
\item If $\{E_j\}_{j=1}^\infty\subseteq\mathscr{M}$ and $E_1\supseteq E_2\supseteq E_3\supseteq\cdots$, then $\mu\left(\bigcap \limits_{j=1}^\infty E_j\right)=\lim\limits_{j\rightarrow\infty}\mu(E_j)$
\end{enumerate}

Proof:

\begin{enumerate}
    \item Note that $\mu(F) = \mu(E \cup E^c) = \mu(E) + \mu(E^c) \geq \mu(E)$.

    \item It suffices to show the property holds with $E_1$ and $E_2$.
        Note that, $E_2 \setminus \{E_1\} \subseteq E_2$, \\so $\mu(E_2 \setminus \{E_1\}) \le \mu(E_2)$. Then, 

        $$\mu(E_1 \cup E_2) = \mu(E_1) + \mu(E_2 \setminus \{E_1\}) \le \mu(E_1) + \mu(E_2)$$.

    \item Let $A = \lim\limits_{j\rightarrow\infty}\mu(E_j)$. Then $\forall \epsilon > 0$, there exists $N \in\mathbb{N}$ such that for all $n \geq N$, $A - \mu(E_n) < \epsilon$.

    Since, for all $i \geq 1$, $E_i \subseteq E_{i+1}$, $\mu(E_n) = \mu(\bigcup\limits_{i=1}^{n}E_i)$.

    Also, by similar reason, $\mu(\bigcup\limits^{\infty} E_i) \geq \mu(\bigcup\limits_{i=1}^{n}E_i)$.

    Finally, $\epsilon > A - \mu(E_n) = A - \mu(\bigcup\limits^{n}E_i) \geq A - \mu(\bigcup\limits^{\infty} E_i)$, so $A = \mu(\bigcup\limits^{\infty} E_i)$.

    \item From part 3, for each $i$, let $F_i$ such that $E_i^c = F_i$.

    Then $F_1\supseteq F_2\supseteq F_3\supseteq\cdots$, and

    $\mu(\bigcap \limits_{j=1}^\infty F_j) = \mu(\mathbb{X}) - \mu((\bigcap \limits_{j=1}^\infty F_j)^c)$

    $\hspace{1.7cm}= \mu(\mathbb{X})-\mu(\bigcup \limits_{j=1}^\infty E_j)$

    $\hspace{1.7cm}= \mu(\mathbb{X}) - \lim\limits_{j\rightarrow\infty}\mu(E_j)$

    $\hspace{1.7cm}= \lim\limits_{j\rightarrow\infty}\mu(E_j^c) $

    $\hspace{1.7cm}= \lim\limits_{j\rightarrow\infty}\mu(F_j)$. $\square$
\end{enumerate}

\end{document}

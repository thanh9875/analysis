\documentclass[12pt]{article}
\usepackage{fullpage}
\usepackage{graphicx}
\usepackage{sidecap}
\usepackage{algorithmic}
\usepackage{algorithm2e}
\usepackage{bbm}
\usepackage{amssymb}
\usepackage{amsmath}
\usepackage{amsfonts}
\usepackage{amsthm}
\usepackage{yhmath}
\usepackage[mathscr]{euscript}
\usepackage{enumerate}
\usepackage[hmargin=1in,vmargin=1in]{geometry}

\begin{document}
Let $m$ be the Lebesgue measure as described in class.  Let $\mathscr{C}$ be the Cantor set as described in class.  Compute $m(\mathscr{C})$.

Proof:

Let $\mathscr{C}$ be the Cantor Set.

Then, with deMorgan's law and Assessment 17 part 3,

$\hspace{1cm}\mu(\mathscr{C})$

$\hspace{0.5cm}= \mu(\bigcap\limits_{i}^{\infty} C_i)$

$\hspace{0.5cm}= 1 - \mu(\bigcup\limits_{i}^{\infty} C_i^c)$

$\hspace{0.5cm}= 1 - \lim\limits_{n \to \infty} \mu(\mathscr{C}_n^c)$

$\hspace{0.5cm}= 1 - \lim\limits_{n \to \infty} \sum\limits_{i=0}^{n} \frac{1}{3} (\frac{2}{3})^i$

$\hspace{0.5cm}= 1 - \frac{1}{3}\times\frac{1}{1-\frac{2}{3}}$

$\hspace{0.5cm}= 1 - 1  = 0$.

Since $\mu$ is a measurement, $\mu({\mathscr{C}}) = 0$. $\square$



\end{document}

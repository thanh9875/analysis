\documentclass[12pt]{article}
\usepackage{fullpage}
\usepackage{graphicx}
\usepackage{sidecap}
\usepackage{algorithmic}
\usepackage{algorithm2e}
\usepackage{bbm}
\usepackage{amssymb}
\usepackage{amsmath}
\usepackage{amsfonts}
\usepackage{amsthm}
\usepackage{yhmath}
\usepackage[mathscr]{euscript}
\usepackage{enumerate}
\usepackage[hmargin=1in,vmargin=1in]{geometry}

\begin{document}
\item For each of the following, decide if the stated function $\mu$ does indeed define a measure on the provided $\mathbb{X}$.  Explain your answer.
\begin{enumerate}
\item $\mathbb{X}\subseteq\mathbb{R}$ nonempty, $\mathscr{M}$ a $\sigma$-algebra.  $\mu:\mathscr{M}\rightarrow[0,\infty]$ by $\mu(E)=$the number of elements in $E$ for all $E\in\mathscr{M}$.
\item $\mathbb{X}\subseteq\mathbb{R}$ nonempty, $x_0\in\mathbb{X}$, $\mathscr{M}$ a $\sigma$-algebra.  $\mu:\mathscr{M}\rightarrow[0,\infty]$ by $\mu(E)=\begin{cases}1&x_0\in E\\0&x_0\nin E\end{cases}$ for all $E\in\mathscr{M}$.
\item $\mathbb{X}\subseteq\mathbb{R}$ uncountable, $\mathscr{M}$ the $\sigma$-algebra $\mathscr{M}=\left\{E\subseteq \mathbb{X}:E\text{ is countable or }E^c\text{ is countable}\right\}$.  $\mu:\mathscr{M}\rightarrow[0,\infty]$ by $\mu(E)=\begin{cases}0&E\text{ is countable}\\1&E^c\text{ is countable}\end{cases}$ for all $E\in\mathscr{M}$.
\item $\mathbb{X}\subseteq\mathbb{R}$ infinite, $\mathscr{M}=\mathscr{P}(\mathbb{X})$.  $\mu:\mathscr{M}\rightarrow[0,\infty]$ by $\mu(E)=\begin{cases}0&E\text{ finite}\\\infty&E\text{ infinite}\end{cases}$ for all $E\in\mathscr{M}$.
\end{enumerate}

Proof

\begin{enumerate}
\item Yes. First $\mu(\emptyset) = 0$. Disjoint sets do not have same elements, so the number of elements of the union of them will equal to the sum of the number of elements, including infinitely many.
\item Yes. First $\mu(\emptyset) = 0$. For any element $x_0$, among all disjoint sets, $x_0$ is inside at most one of them. So the measurement of the union will be either 1, if one of them contains $x_0$, or 0, if none of them contains $x_0$. And that is equal to either 1 or 0 in both cases.
\item Yes. First $\mu(\emptyset) = 0$.
      For 2 countable sets, the measurement of the union is $0$. The sum of measurement is $0$.

      For 2 uncountable sets, the measure of the union is $1$. The sum of measurement is $1$.

      For a countable and an uncountable, the measure of the union is $1$. The sum of measurement is $1$.
\item No. For $n \in \mathbb{N}$, let $E_n = \{n\}$. The union of $E_n$ is infinite. Its measurement is $\infty$, and the sum of each $E_n$ is $0$.
\end{enumerate}
$\square$
\end{document}

\documentclass[12pt]{article}
\usepackage{fullpage}
\usepackage{graphicx}
\usepackage{sidecap}
\usepackage{algorithmic}
\usepackage{algorithm2e}
\usepackage{bbm}
\usepackage{amssymb}
\usepackage{amsmath}
\usepackage{amsfonts}
\usepackage{amsthm}
\usepackage{yhmath}
\usepackage{siunitx}
\usepackage[mathscr]{euscript}
\usepackage{enumerate}
\usepackage{mathtools}
\usepackage[hmargin=1in,vmargin=1in]{geometry}
\usepackage{graphicx}
\usepackage{subfigure}
\usepackage{setspace}
\usepackage{systeme}
\linespread{1.4142}
\graphicspath{ {./images/} }

\newcommand{\Z}{\mathbb{Z}}
\newcommand{\Q}{\mathbb{Q}}
\newcommand{\R}{\mathbb{R}}
\newcommand{\C}{\mathbb{C}}
\newcommand{\curve}{\mathcal{C}}


\begin{document}
\begin{enumerate}
    \item $|Inn($G$)|$ = 1 if and only if $G$ is Abelian.

        Suppose $|Inn($G$)|$ = 1. We know that $\phi_e \in \text(Inn($G$))$, so for any $g \in G$, $\phi_g = \phi_e$. This means that, for any $x \in G$, $\phi_g(x) = gxg^{-1} = \phi_e(x) = x$. Reduce further yields $gx = xg$ for any $g$ and $x$ in $G$. So $G$ is Abelian.

        Suppose $G$ is Abelian. Let $g \in G$. Consider $\phi_g \in Inn(G)$. For any $x \in G$, $\phi_g(x) = gxg^{-1} = gg^{-1}x = x$. So, for any $g\in G$, $\phi_g$ is the identity innner isomorphism. Thus $|Inn(G)| = 1$.
        \pagebreak

    \item Show that a group with more than one subgroup of order 5 must have order of at least 25.

        Let $H$ and $K$ be subgroups of order 5 of $G$.
        Consider $H \cap K$. For any element in $H \cap K$, that element belongs to both $H$ and $K$. Since $H$ and $K$ are groups, they both contains the inverse of that element, and by extension, $H \cap K$ also contains that inverse. We also know that $e \in H\cap K$. So $H \cap K$ is a subgroup of $H$ and $K$.

        By Lagrange's theorem, $|H\cap K|$ is a divisor of $5$. If $|H\cap K| = 5$, they are the same group, so $|H \cap K| = 1$.
        Finally, any element of $HK$ is an element of $G$ because of closure, $|G| \geq |HK| = \frac{|H||K|}{|H\cap K|} = 25$.
        \pagebreak

    \item Let $G = U(16), H = \{1, 15\}, K = \{1,9\}$. Determine with proof whether $H \cong K$ and $G/H \cong G/K$.

        Consider the mapping $f: H \to K$ such that $1 \overset{f}{\mapsto} 1$ and $15 \overset{f}{\mapsto} 9$. Clearly $f$ is well-defined and bijective.
        And
        \begin{align*}
            f(1 \times 1) &= f(1) = 1 = f(1) \times f(1) \\
            f(1 \times 15) &= f(15) = 9 = f(1) \times f(15)\\
            f(15 \times 15) &= f(1) = 1 = f(15) \times f(15)
        \end{align*}

        So, $H \cong K$.

        Assume that $G/H \cong G/K$, then there exists a function $f: G/H \to G/K$ that is bijective and operation-preserving.

        Consider $1 \times \{1,15 \}$ and $3 \times \{1,15\}$ of $G/H$.

        Note that,
        \begin{align*}
            (1\times\{1,15\})^2 &= 1 \times \{1,15\} = \{1,15\}\\
            (3\times\{1,15\})^2 &= 9 \times \{1,15\} = \{7,9\}
        \end{align*}

        The set $G/K$ contains all of the following:
        \begin{align*}
            1 \times \{1,9\} &= \{1,9\} = 9 \times \{1,9\}\\
            3 \times \{1,9\} &= \{3,11\} = 11 \times \{1,9\}\\
            5 \times \{1,9\} &= \{5,13\} = 13 \times \{1,9\}\\
            7 \times \{1,9\} &= \{7,15\} = 15 \times \{1,9\}
        \end{align*}

        Note that, $(1\times \{1,9\})^2 = (3\times \{1,9\})^2 = (5 \times \{1,9\})^2 = (7 \times \{1,9\})^2 = \{1,9\}$. Then, since $f$ is bijective and operation-preserving, $\forall x, y \in G/H$, $x^2 = y^2$, which is a contradiction.

        So, $G/H \not \cong G/K$.
        \pagebreak

    \item (Easy)(Chapter 10, Exercise 8)

        Let $sgn(\sigma)$ be a function such that, $$sgn: S_n \to (\{-1, 1\}, \times); sgn(\sigma) \mapsto \begin{cases} 1 & \sigma \text{ is even} \\ -1 & \sigma \text{ is odd}\end{cases}$$

        Show that $sgn$ is a homomorphism, and find $Ker(sgn)$.

        Clearly, $(\{1,-1\}, \times)$ is a group.
        For any 2 permuation in $S_n$, one of the following must holds:
        \begin{align*}
            &sgn(even \times even) = sgn(even) = 1 = 1 \times 1 = sgn(even) \times sgn(even)\\
            &sgn(odd \times even) = sgn(odd) = -1 = -1 \times 1 = sgn(ood) \times sgn(even)\\
            &sgn(odd \times odd) = sgn(even) = 1 = -1 \times -1 = sgn(odd) \times sgn(odd)
        \end{align*}
        So, $sgn$ is a homomorphism. Since $1$ is the identity, the $Ker(sgn)$ is $A_n$.
        \pagebreak

    \item (Moderate)(Chaper 7, Exercise 48)

        Let $G$ be a group and $|G| = pqr$ where $p,q,r$ are prime numbers. Let $H$ and $K$ be subgroup of $G$ such that $|H| = pq$ and $|K| = qr$. Show that $|H \cap K| = q$.

        Proof:

        Since $H \cap K$ is a subgroup of $H$ and $K$ (see Problem 2), $|H \cap K|$ is a common divisor of $|H|$ and $|K|$. Since $p,q,r$ are prime number, common divisors of $pq$ and $qr$ are $1$ and $q$.

        Note that, similar to Problem 2, $pqr = |G| \geq |HK| = \frac{|H||K|}{|H \cap K|} = \frac{pq^2r}{|H \cap K|}$,\\ so $|H \cap K| \geq \frac{pq^2r}{pqr} = q$.

        So $|H \cap K| = q$.
        \pagebreak

    \item (Moderate)(Chapter 6, Exercise 66)

        Show that $(\Q \setminus \{0\} , \times)$ and $(\Q, +)$ are not isomorphic.

        Proof:

        Suppose there exists $f:(\Q \setminus \{0\} , \times) \to (\Q, +)$ such that $f$ is bijective and operation-preserving.

        Let $p_0 \in \Q \setminus \{0\}$ such that $f(p_0) = 0$. For any $x \in(\Q \setminus \{0\} , \times)$, \\$f(p_0 \times x ) = f(p_0) + f(x) = f(x)$. Since $f$ is injective, $p_0 \times x = x$, so $p_0 = 1$.

        Consider $f(-1 \times -1) = f(-1) + f(-1)$. The left hand side $f(-1 \times -1) = f(1) = 0$, while the right hand side is $2f(-1)$, so $f(-1) = 0$, which is a contradiction since $f$ is injective.

        So $(\Q\setminus\{0\}, \times)$ is not isomorphic to $(\Q, +)$.
        \pagebreak

    \item (Moderate)(Chapter 9, Exercise 9)

        Show that a subgroup of index 2 is normal.

        Proof:

        Let $G$ be a group and $H$ is a subgroup of index 2.

        Let $g \in G$. If $g \in H$, $gH = H = Hg$, so $H$ is normal.

        If $g \not \in H$, $gH$ and $H$ are the two disjoint left coset of $H$ in $G$. So, $gH = G - H$. Similar argument can be made for disjoint right coset, so $Hg = G - H$. \\Thus, $gH = G - H = Hg$.

        In either case, $H$ is normal.
        \pagebreak

    \item (Hard)(MIT Putnam Seminar Fall 2018, Abstract Algebra, Problem 17)

        Show that a group of order $4n+2$ for $n \in \mathbb{N}$ has a proper normal subgroup.

        Proof:

        Let $G$ be a group of order $4n+2$, $n \geq 1$. By Cayley's Theorem, $G$ is isomorphic to a subgroup of $S_{4n+2}$.

        Let $f: G \to \bar G$ such that $\bar G$ is a subgroup of $S_{4n+2}$ and $f$ satisfies the isomorphism properties.

        By Cauchy's Theorem, there exists an element $g \in G$ such that $|g| = 2$.

        Note that, $f(g)$ is a permuation in $S_{4n+2}$ and can be written as the product of disjoint cycles. Also, since $|g| = 2$, $g = g^{-1}$, thus $f(g) = f(g)^{-1}$.

        So $|f(g)| = 2$. Then $f(g)f(g)^{-1} = e$, and since $e$ can be written as product of $4n+2$ disjoint cycles, and $f(g)$ and $f(g)^{-1}$ have the same length, $f(g)$ can be written as the product of $\frac{4n+2}{2} = 2n+1$ disjoint cycles. Thus, $f(g)$ is odd.

        In $f(G)$, for any even permuation, the product of $f(g)$ with that cycle is odd. So the number of odd and even permuation is the same and equal to $2n+1$.

        Clearly, the set of all even permuation in this group is a proper subgroup by One-step Subgroup test. For any permuation, its coset with this subgroup is either exactly this subgroup (when the permutation is even) or the set of odd permutations. So the index of this subgroup is $2$, and by Problem 7, it is normal.

        Thus, its pre-image in $G$ is proper and normal.

        $\square$
        \pagebreak

\end{enumerate}

\end{document}

\begin{comment}

to delete files: latexmk -c

to compile: pdflatex (filename).tex

to add a pdf
\begin{figure}[!htbp]
    \centering
    \includegraphics[width=1 \linewidth]{{filename}.pdf}
    \caption{Data sheet}
\end{figure}

\end{comment}
